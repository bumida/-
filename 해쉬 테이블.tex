# 해쉬 테이블(Hash Table)
1. 해쉬 구조
Hash Table : 키(Key)에 데이터(Value)를 저장하는 데이터 구조
Key를 통해 바로 데이터를 받아올 수 있으므로, 속도가 획기적으로 빨라짐
파이썬 딕셔너리(Dictionary) 타입이 해쉬 테이블의 예: Key를 가지고 바로 데이터(Value)를 꺼냄
보통 배열로 미리 Hash Table 사이즈만큼 생성 후에 사용(공간과 탐색 시간을 맞바꾸는 기법)
단, 파이썬에서는 해쉬를 별도 구현할 이유가 없음 - 딕셔너리 타입을 사용하면 됨

# hash table 만들기

hash_table =list([0 for i in range(10)])
hash_table

[0,0,0,0,0,0,0,0,0,0]

이번엔 초간단 해쉬 함스를 만들어봅시다.
다양한 해쉬 함수 고안 기법이 있으며, 가장 간단한 방식이 Division 법

def hash_func(key):
	return key % 5
	
data1 = 'Andy'
data2 = 'Dave'
data3 = 'Trump'

## ord(): 문자의 ASCII(아스키) 코드 리턴
print(ord(data1[0]), ord(data2[0]), ord(data3[0]))

해쉬 테이블에 값 저장 예
data:value 와 같이 data 와 value를 넣으면, 해당 data에 대한 key를 찾아서, 해당 key에 대응하는 해쉬주소에 value를 저장하는 예

def storage_data(data, value):
key = ord(data[0])
hash_address = hash_func(key)
hash_table[hash_address] = value

실제 데이터를 저장하고, 읽어보겠습니다.
def get_data(data):
  key = ord(data[0])
  hash_address = hash_func(key)
  hash_table[hash_address] = value
  
자료 구조 해쉬 테이블의 장단점과 주요 용도

장점
- 데이터 저장/읽기 속도가 빠르다.(검색 속도가 빠르다.)
- 해쉬는 키에 대한 데이터가 있는지(중복) 확인이 쉬움
단점
- 일반적으로 저장공간이 좀더 많이 필요하다.
- 여러 키에 해당하는 주소가 동일한 경우 충돌을 해결하기 위한 별도 자료구조가 필요함.
주요 용도
- 검색이 많이 필요한 경우
- 저장, 삭제, 읽기가 빈번한 경우
- 캐쉬 구현시 (중복 확인이 쉽기 때문)

파이썬에서의 hash table의 활용 예 - 딕셔너리(Dictionary)

- 내부적으로는 결국 hash로 구현
- 예: 스마트폰에 전화번호 저장하기
	전화번호 저장시 이름을 저장하기
	이름: 여친, 전화번호: 000-2222-3333
	이름으로 전화번호 찾기
	
	
  프로그래밍 연습
  
  연습1 : 리스트 변수를 활용해서 해쉬 테이블 구현해보기
  1. 해쉬 함수: key % 8
  2. 해쉬 키 생성: hash(data)
  
  hash_table = list([0 for i in range(8)])
  
  def get_key(data):
	return hash(data)
	
  def hash_function(key):
    return key % 8
	
  def save_data(data, value):
	hash_address = hash_function(get_key(data))
	hash_table[hash_address] = value
	
  def read_data(data):
    hash_address = hash_function(get_key(data))
	return hash_table[hash_address]
	

  

  
  
  